\chapter*{Acknowledgment}
\addcontentsline{toc}{chapter}{Acknowledgment}
\begin{chapquote}{Carl Friedrich Gauss}
It is not knowledge, but the act of learning, not possession but the act of getting there, which grants the greatest enjoyment
\end{chapquote}

Scientific work such as the doctoral thesis at hand would not be possible without the support of several people.
I was fortunate enough to meet a number of great people during the preparation of this thesis I would like to express my gratitude to.
First and foremost, I would like to thank my supervisor, Prof. J\"org Conradt, who gave me opportunity to work on this interesting topic while giving me the freedom to creatively explore the research area.
Furthermore, he introduced me to his \acl{NST} at Technical University of Munich, where he created an atmosphere of creativity, scientific rigor but also great camaraderie.
Randomizing the order, I would like to thank Cristian Axenie for his enthusiasm, encouragement as well as endless nights at the robot lab, Nikolai Waniek for discussions on scientific as well as any other topics of current interest, Christoph Richter for proof-reading papers as well as his pragmatic approach to research, Lukas Everding for his deadpan sense of humor and Eme\c{c} Er\c{c}elik for his help and support with organizing practical lectures at the University as well as at the BMW Summer School.
Prof. Conradt also introduced me to the neuromorphic engineering community in general and to Dr. Terry Stewart from \ac{CNRG} at University of Waterloo in particular, who also had a great impact on my doctoral research.
I would like to thank him for his continuous support across the Atlantic Ocean and for generously sharing his wide knowledge on spiking neurons, Nengo, and machine learning in general, but moreover for becoming a great mentor and role model for me.
Furthermore, I am also deeply grateful to Prof. Chris Eliasmith, the head of \ac{CNRG}, as well as Terry and all the members of \ac{CNRG} for hosting me during a six week research visit at University of Waterloo in summer 2017.
This visit not only allowed me to push my research forward thanks to the remoteness of being an ocean and a six hour time delay apart from the \enquote{usual business}.
Furthermore, I was also fortunate to meet lots of great people at University of Waterloo sharing their knowledge and experience, but I also had a great time at jam sessions or simply visiting and enjoying the beauty of Canada.
My research visit also allowed me to initiate a collaboration project between \ac{ABR}, a \ac{CNRG} spin-off company, and BMW Group with some results of that project making their way into this thesis.
As part of this collaboration project, I would like to thank Terry, Peter Blouw, Daniel Rasmussen and Eric Hunsberger on the technical side, but also Peter Suma on the administrative side for fighting with me through the pain of paperwork to make the project actually work.

I am also deeply thankful to Dr. Hans-J\"org V\"ogel, my supervisor at BMW Group, for giving me the opportunity to pursue my Ph.D. in the environment of BMW's research department.
Furthermore, he always supported and advised me with scientific and administrative tasks, defended the idea of applying neuromorphic computing \enquote{in a car} within the company and always had an \enquote{open door}, whenever something unexpected occurred.
I would also like to thank Martin Arend, the leader of my department for the most time during my Ph.D. years at BMW Group for being the nicest boss a Ph.D. student could hope for and furthermore, for not only supporting my research visit to Canada, but also joining for a couple of days and giving me the opportunity to accompany him when meeting the local start-up scene.
Another great part of pursuing my doctoral studies at BMW Group, was the network of fellow Ph.D. students within BMW's ProMotion program.
Meeting and talking to other students with similar tasks, success stories but also problems and frustrations helped me to overcome these usual but unpleasant parts of doctoral studies when progress stagnated.
In particular, I would like to express my gratitude towards Leopold Walkling and Julian Tatsch for sharing their knowledge, experience and data sets.
Furthermore, I would like to thank the \enquote{Improve} group (in no particular order): Franziska Hertlein, Sascha Steyer, Nicola Hupp, Florian Roider, Jens Schulz, Marc Vogt, Daniel Knobloch, Alexander Terres, Jan Korus, Julius Riedelbauch, Michael Ponnath, Peter R\"osch, Philip Kotter and Annette B\"ohmer.
I would also like to thank my fellow Ph.D. students at our department LT-3 Maike Hartstern and Christoph Segler for sharing the ups and downs of a Ph.D. at BMW in general and at LT-3 in particular as well as interesting scientific and daily-life discussions.
Particularly, I would like to thank Christoph for our shared efforts regarding the \enquote{management} of the dev-box as well as remembering literally every single administrative problem one could run into at BMW and, more importantly, its solution.
I would also like to express my gratitude to a couple of \enquote{regular} BMW colleagues for giving feedback to papers, talks or slides, helping out with administrative issues, interesting lunch-break discussions or simply for being great colleagues.
Again, randomizing the order, I would like to thank Mohsen Kaboli, Tom Hubregtsen, Sebastian Wirkert, Fridolin Bauer, Nadine Stockinger and Suomy Jacob. 
Additionally, I would like to thank Angelika Sch\"afer and Simone Barnloher as well as Susanne Schneider for their help and patience regarding administrative tasks related to business trips at BMW or student supervision at TUM respectively.
Finally, I would like to thank all the students I had the pleasure to supervise either at BMW or at TUM doing master or bachelor theses, project practicals, scientific seminars or simply lectures.
I hope they have learned as much from me as I learned from them.

I am also deeply thankful to my parents, Rainer and Ingrid Mirus, for being a constant source of support and encouragement when things do not work as originally planned and for providing a basement to partly finish writing of this thesis.
Thank you for supporting my life-long journey from the first steps up until this point and beyond.
Furthermore, I would like to express my gratitude to Helga Wallitzer, my mother-in-law, especially for her support during the first weeks after the birth of our daughter Isabella, which allowed me to find some sleep and get at least some work done during that great yet stressful time.
Last but not least, I would like to express the most heartfelt gratitude to my lovely wife Lisa for being on my side ever since I made the decision to move to Munich for the Ph.D., enduring my temper during the frustrating times, supporting me selflessly in everything that I do and for making me the greatest gift in the world: our daughter Isabella.
I could not have finished this thesis without you.
Thank you for everything! 

\vspace{1cm}
\fullname\\
Munich, September 2019
