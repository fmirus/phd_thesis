\chapter*{Zusammenfassung}
\addcontentsline{toc}{chapter}{Zusammenfassung}

Der Wettlauf in Richtung des ersten vollkommen autonom fahrenden Automobils ist aktuell eine der Hauptursachen f\"ur den rasanten Fortschritt der Forschung im Automobilsektor.
Ein Hauptgrund f\"ur diese Entwicklung in den letzten Jahren ist der Fortschritt im Bereich der k\"unstlichen Intelligenz und insbesondere von sogenannten tiefen neuronalen Netzen, die au{\ss}ergew\"ohnliche Erfolge in Bereichen erzielt haben, die f\"ur das autonome Fahren von entscheidender Bedeutung sind.
Der Schwerpunkt des aktuell noch jungen, doch wachsenden Forschungsgebietes \emph{Neuromorphic Computing} liegt hingegen auf biologisch inspirierten Computersystemen und Algorithmen, die darauf abzielen das Gef\"alle zwischen biologischen und k\"unstlichen Systemen hinsichtlich Leistung und Effizienz zu verringern.
Obwohl neuromorphe Hardware-Prototypen technologisch noch nicht so ausgereift sind wie Ihre konventionellen Gegenst\"ucke, so zeigen sie doch vielversprechendes Potenzial zuk\"unftige Fahrzeuge im Hinblick auf Energieffizienz zu verbessern.
Allerdings ziehen diese neuromorphen Ans\"atze nur langsam das Interesse der Automobilindustrie auf sich, da diese neuartigen Computerarchitekturen auf sogenannten gepulsten neuronalen Netzen (kurz \acsp{SNN} vom englischen \aclp{SNN}) basieren, welche sich drastisch von klassischen neuronalen Netzen unterscheiden und daher neuartige Paradigmen hinsichtlich Algorithmik und Programmierung ben\"otigen werden.

In dieser Dissertation machen wir einen ersten Schritt in Richtung eines kognitiven Umgebungsmodells f\"ur Anwendungen im Bereich des automatisierten Fahrens unter Verwendund verteilter Darstellungen und gepulster neuronaler Netze.
Wir untersuchen den Einsatz von Vektor-Darstellungen, die bisher vorrangig f\"ur die Modellierung kognitiver Prozesse oder in der Sprachverarbeitung verwendet wurden, zur Wissensrepr\"asentation im Fahrzeugumfeld.
Dieser generische Ansatz zur Informationsdarstellung erlaubt den Einsatz in verschiedenen Anwendungsbereichen ohne gro{\ss}e an der Repr\"asentation selbst.
Dar\"uber hinaus eignet sich eine solche Darstellung f\"ur die Implementierung in der Sprache von gepulsten neuronalen Netzen, welche neuartige, effiziente Lernverfahren sowie die Verwendung mit didizierter neuromorpher Hardware erlaubt.
Weiterhin lassen sich dadurch die Vorteile symbolischer Darstellungen mit den St\"arken und automatisierten Lernverfahren von neuronalen Netzen kombinieren.
Wir untersuchen die Eignung unserer vektor-basierten Szenen-Repr\"asentation an Hand von verschiedenen Anwendungsbeispielen.
In einer ersten Anwendung pr\"asentieren wir ein Modell, welches basierend auf unserer Vektor-Darstellung der Umgebung lernt den aktuellen Fahrkontext zu klassifizieren.
Der konzeptionelle Fokus liegt dabei darauf, die semantische Bedeutung der aktuellen Fahrsituation in der Darstellung zu erfassen, aber auch den Einfluss unterschiedlicher Faktoren wie beispielsweise des zugrundeliegenden Vektor-Vokabulars oder des verwendeten Lernverfahrens zu untersuchen.
Ein weiterer wichtiger Bestandteil eines Umgebunsmodells im Automobilbereich ist genaues Wissen \"uber den aktuellen Zustand sowie die Vorhersage aller sich bewegenden Objekte im Umfeld des Fahrzeugs.
Wir untersuchen die Pr\"adiktion dieser Objekte auf Basis unserer verteilten Repr\"asentation.
Diese nutzt eine neuartige Vektor-Darstellung r\"aumlicher Informationen unter Verwedung von Vektor-Potenzen, welche auf der zyklischen Faltung basieren.
Hier untersuchen wir die Hypothese, dass eine solche strukturierte Darstellung in der Lage ist den gegenseitigen Einfluss zwischen den verschiedenen, sich bewegenden Verkehrsteilnehmern zu erfassen.
Weiterhin erlaubt dieses herausfordernde Anwendungsbeispiel die Untersuchung verschiedener Lernverfahren.
Ein erfahrener Autofahrer hat im Laufe der Zeit ein umfassendes Verst\"andnis daf\"ur entwickelt, wie sich andere Verkehrsteilnehmer in bestimmten Situationen verhalten werden.
Gleichzeitig ist er aber auch in der Lage, dieses Vorwissen auf Basis der aktuellen Fahrsituation kontinuierlich anzupassen und sich auf das Verhalten der anderen Fahrer spontan einzustellen.
Mit dieser Inspiration entwickeln wir ein generisches Modell zur Vorhersage von dynamischen Objekten, welches auf Basis von vorab gesammelten Daten lernt.
Dieses Modell verfeinern wir durch ein sogenanntes Abstimmen-von-Experten-Modell (englisch mixture-of-experts), welches in der Lage ist die Bewegungsvorhersage zur Laufzeit auf Basis des aktuellen Fahrkontexts anzupassen.
Erg\"anzend pr\"asentieren wir eine neuartige neuromorphe Steuerungsarchitektur, die die Implementierung von Steuerungsalgorithmen in der Sprache gepulster neuronaler Netze erlaubt.
Dies erm\"oglicht durch die Aufteilung in kleiner Teilnetze die Vorteile manueller Programmierung mit automatisierten Lernverfahren neuronaler Netze zu kombinieren.
Dadurch gewinnen wir einen ersten Eindruck, wie eine zuk\"unftige, neuronale Fahrzeugsteuerung auf Basis unseres kognitiven Umgebungsmodells aussehen kann.
