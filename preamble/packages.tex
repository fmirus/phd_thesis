\usepackage{latexsym}         % Fuer recht seltene Zeichen
\usepackage[utf8]{inputenc}   % =E4 =F6 =FC =DF; danach  geht auch das ß richtig
\usepackage{caption}          % Figure-Captions formatieren
%\captionsetup[subfigure]{font={sf,md,sl,small},labelfont={sf,md,sl,small}}
%\setkomafont{caption}{\itshape\sffamily}
%\setkomafont{captionlabel}{\upshape\bfseries\sffamily}
%\usepackage{sectsty}          % Section headings formatieren
\usepackage{amsmath,amsthm}
\usepackage{mathdots}
%\usepackage{fancyhdr}
\usepackage[automark, komastyle, headsepline, plainfootsepline]{scrpage2}
\usepackage{amssymb}
\usepackage{xfrac}
\usepackage{float}
\usepackage{algorithm}
\usepackage{algorithmic}
\usepackage[a4paper,lmargin={2.5cm},rmargin={2.5cm},tmargin={3cm},bmargin={2.5cm},marginparwidth=40pt]{geometry}
\usepackage{picinpar}
\usepackage{lipsum}
\usepackage{nicefrac}
\usepackage{enumitem}
\usepackage{graphicx}
\usepackage{wrapfig}
\usepackage{graphics} % for pdf, bitmapped graphics files
\usepackage{epsfig} % for postscript graphics files
\usepackage{mathptmx} % assumes new font selection scheme installed
\usepackage{times} % assumes new font selection scheme installed
\usepackage{amsmath} % assumes amsmath package installed
\usepackage{amssymb}  % assumes amsmath package installed
\usepackage{inputenc}
\usepackage{todonotes}
\usepackage{marginnote}
\usepackage{makecell}
\usepackage[binary-units=true]{siunitx}
\DeclareSIUnit{\nothing}{\relax}
\usepackage{xcolor}
\usepackage{eqparbox}
%\usepackage{subcaption}
\usepackage{scrhack}
\usepackage{epigraph}
\usepackage[caption=false,font=footnotesize]{subfig}
\usepackage[acronym, toc, shortcuts,subentrycounter]{glossaries}
\usepackage{nomencl}
\usepackage{tumcolor}
\usepackage{booktabs}
\usepackage{multirow}
\usepackage{makecell}
\renewcommand\theadalign{bc}
\renewcommand\theadfont{\bfseries}
\renewcommand\theadgape{\Gape[4pt]}
\renewcommand\cellgape{\Gape[4pt]}
%\usepackage[]{isodate}
\usepackage[backend=biber,
            style=authoryear,
            %style=alphabetic,
            %style=numeric,
			sorting=anyt,
            natbib=true,
			urldate=iso8601,
			date=iso8601,
			giveninits=true,
			maxbibnames=99,
            uniquelist=false,
			maxcitenames=2]{biblatex}
\addbibresource{../literature/literature.bib}
\usepackage{stmaryrd} % for some symbols like \varoast (* within a circle)
\usepackage{url}
\def\UrlBreaks{\do\/\do-}
\setcounter{biburlnumpenalty}{100}
\setcounter{biburllcpenalty}{100}
\setcounter{biburlucpenalty}{100}
\usepackage{bm}
\usepackage{tikz}
\usepackage{tikz-network}
\usetikzlibrary{calc}
\usetikzlibrary{positioning}% To get more advanced positioning options
\usetikzlibrary{arrows}% To get more arrow heads
\usetikzlibrary{fit}
\usetikzlibrary{graphs}
\usetikzlibrary{quotes}
\usetikzlibrary{shapes.geometric}
\usetikzlibrary{backgrounds}
\usepackage[
  pdftex,
	breaklinks,
	colorlinks=true,
	linkcolor=black,
  citecolor={blue!50!black},
  urlcolor={blue!80!black}
	% linkcolor={red!50!black},
%	urlcolor=black,
%	citecolor=black
	]{hyperref}
