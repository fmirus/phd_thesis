\chapter{Related Work}
This chapter aims to briefly present the relevant work related to the thesis at hand.
This thesis touches several diverse research disciplines like neuroscience, autonomous driving and robotics, machine learning and neuromorphic computing, which need to be reviewed.

\section{Autonomous Driving and Mobile Robotics}
In order to navigate safely to a desired goal, a mobile agent needs to solve several problems like localization ("where am I?"), path planning ("which way do I want to go?"), environment perception ("what is around me?"), knowledge representation and reasoning ("which decisions to infer from available information?") as well as motion control ("how to move my actuators?").
Mobile robotics is the science of building computer-controlled mechanical devises, which are able to tackle these issues autonomously.\\
Autonomous driving in automotive context is a subfield of this area, since an autonomous vehicle is nothing more than a wheeled mobile robot, which is able to fulfil the transportation capabilities of a traditional car without human input. 
Autonomous driving is currently a major research topic, since a fully autonomous vehicle, which is able to tackle challenging driving situations without external input and to rival human performance, is yet to be build.\\
On the road to fully automated driving, several \ac{ADAS} have been developed during the last decade and thus made a huge jump by incrementally increasing complexity and therefor the level of autonomy. 
From the first grand \cite{Thrun2006} and urban challenges \cite{Urmson.2008}, organized by the \ac{DARPA}, to recent experiments of Google\footnote{\url{https://www.google.com/selfdrivingcar}} in the field of autonomous driving, \ac{ADAS} have made their way into series-production vehicles bringing the potential to increase comfort and safety in road traffic in the long run. 
There exists a large variety of commercial systems, like e.g. \ac{ACC} or intelligent parking assistance systems, modern vehicles are already equipped with.\\
As this thesis focuses on environmental modelling in context of autonomous driving, this section presents research from this field like road detection and modelling (Sec. \ref{subsec:lane}), object detection (Sec. \ref{subsec:obj_detect}) and tracking (Sec. \ref{subsec:obj_track}) as well as sensor fusion (Sec. \ref{subsec:sensor_fusion}) while other aspects like localization \cite{Levinson2010}, path planning (\todo{citation}) and motion control (\todo{citation}) are neglected.
\subsection{Road Lane Detection and Modelling}
\label{subsec:lane}
\subsection{Object Detection and Classification}
\label{subsec:obj_detect}
\subsection{Object Tracking}
\label{subsec:obj_track}
\subsection{Sensor Fusion}
\label{subsec:sensor_fusion}
\subsection{Low-Level Sensor Fusion}
\section{Machine Learning}
Machine learning is the science of constructing computer programs, which improve with experience. 
This is attractive if manually programming a desired functionality is impossible, intractable or simply not cost-efficient.  
The overall goal of machine learning is to generalize beyond examples, i.e. to generate models that describe the presented input sufficiently well to make the best possible prediction when confronted with previously unseen data.
A formal definition of machine learning has been presented by Thomas M. Mitchell in \cite{Mitchell1997}:

\begin{defn}
A computer program is said to \textbf{learn} from experience $E$ with respect to some class of tasks $T$ and performance measure $P$ if its performance at tasks in $T$, as measured by $P$, improves with experience $E$.
\end{defn}

A large body of research has focused on machine learning last two decades
Through availability of bigger datasets and increased computational power, machine learning has seen some breakthrough advantages in recent years.

There are several methods 
\section{Neural Modelling}
\section{Neuromorphic Systems}
\section{Traditional Computing}